\chapter{Podsumowanie}
W trakcie realizacji niniejszej pracy zapoznano się z technologiami używanymi powszechnie w przemyśle informatycznym przez firmy produkujące oprogramowanie. Dzięki użyciu najpopularniejszych technologii zapewniona była możliwość znalezienia rozwiązania problemów napotkanych w trakcie pisania pracy w internecie.

Aplikacja, która stała się produktem pracy zostanie przekazana do eksploatacji firmie "JN-Profil- badania archeologiczne i historyczne" i będzie dla niej wyraźną pomocą w generowaniu dokumentacji archeologicznej. Przyspieszy to czas jej tworzenia, co wpłynie na zmniejszenie kosztów. 

Ponieważ nie udało się zrealizować wszystkich zakładanych raportów, a jedynie część z nich, dalsza przyszłość rozwoju będzie pod znakiem implementacji brakujących elementów.

Pomimo tego, że dziedzina problemu raczej nie powinna ulegać zmianie, to istnieje prawdopodobieństwo, że dokumentacja archeologiczna może wymagać dokumentów, których wygenerowanie nie zostało zapewnione przez system stworzony w ramach niniejszej pracy inżynierskiej. Jeżeli zajdzie potrzeba, system jest w łatwy sposób rozszerzalny i jego rozwój jest jak najbardziej planowany w przyszłości.

Dodatkowymi produktami powstałymi w trakcie pisania niniejszej pracy są komponenty CRUDTable, ForeignField, DefaultForm oraz szkielet aplikacji. Wszystkie komponenty w niedługiej przyszłości zostaną zgłoszone do twórców szkieletu aplikacji jako kandydat na wtyczke ułatwiającą pracę programistom, którzy wybrali pracę z Vaadinem.
% ex: set tabstop=4 shiftwidth=4 softtabstop=4 noexpandtab fileformat=unix filetype=tex spelllang=pl,en spell: