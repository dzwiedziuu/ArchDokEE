\chapter{Wstęp}
W ostatnich latach, dzięki dynamicznemu rozwojowi informatyki, cyfryzacja jest widoczna w coraz to większej ilości gałęzi gospodarki. Nowoczesne przedsiębiorstwa zaczynają widzieć znaczne oszczędności dzięki wprowadzaniu do codziennej pracy systemów informatycznych. Szczególnym przykładem czynności, które mogłyby być a nawet powinny zostać zautomatyzowane są prace związane z wypełnianiem dokumentów, w których informacje wprowadzane przez pracownika są powielane w wielu miejscach. W bardziej popularnych branżach, rozwiązania usprawniające taką pracę są coraz częściej używane, jednak są także profile działalności, w których informatyzacja byłaby pożądana, ale nie jest wprowadzana.

Jednym z przykładów działalności, gdzie systemy informatyczne usprawniłyby pracę, jest firma prowadząca badania archeologiczne. Zwykłe prace na stanowisku archeologicznym sprowadzają się do fizycznego przebadania pewnego obszaru, a następnie stworzenia dokumentacji podsumowującej wykonane prace, czyli opisującej obiekty archeologiczne oraz zabytki, które zostały znalezione w trakcie badań.

W ostatnich latach, polski rynek badań archeologicznych staje się coraz większy, dzięki inwestycjom związanym z Unią Europejską. Badania archeologiczne prowadzone są na coraz większym obszarze, a co za tym idzie budżety firm archeologicznych są coraz większe. Niestety, wraz ze wzrostem skali badań, wzrasta też ilość wykonywanej dokumentacji. Sytuację komplikuje fakt, że w różnych województwach wymagania są nieco inne. Dodatkowo, wraz ze wzrostem nakładów finansowych inwestowanych w badania archeologiczne zwiększyła się także liczba przedsiębiorstw, które rywalizują ze sobą w przetargach. Większa konkurencja wymusza oczywiście spadek stawek za przeprowadzenie badań, dlatego firmy archeologiczne zmuszone są szczególnie do szukania oszczędności, których może dostarczyć cyfryzacja.

Na szczęście wraz z rozwojem informatyki idzie rozwój narzędzi do tworzenia systemów informatycznych. Szczególną gałęzią informatyki, która dzięki rozwojowi internetu zyskała na znaczeniu, są technologie tworzenia aplikacji internetowych. Jedną z nich, która w niniejszej pracy została wybrana do stworzenia systemu, jest szkielet aplikacji Vaadin, który umożliwia szybkie tworzenie stron WWW. 
\newpage
\section{Cele i zakres pracy}
Głównym celem niniejszej pracy jest stworzenie systemu dla firmy "JN-Profil- badania archeologiczne i historyczne"  wspierającego ją w dokumentowaniu badań archeologicznych. Drugim celem jest zapoznanie ze szkieletem aplikacji Vaadin, ułatwiającym tworzenie aplikacji internetowych. Dodatkowym, choć nieobowiązkowym celem jest stworzenie komponentów graficznych ułatwiających wyświetlanie list obiektów i ich cech a także usprawniających tworzenie formularzy. 

System ewidencji zabytków archeologicznych tworzony w ramach pracy powinien umożliwiać wprowadzanie danych i generowanie dokumentacji archeologicznej w jak najbardziej przystępny i intuicyjny sposób. Aplikacja powinna także umożliwiać wprowadzanie danych w dowolnym miejscu, ze względu na charakter działalności firmy archeologicznej - praca w różnych miejscach kraju.
\section{Układ pracy}
Rozdział 2. zawiera opis systemu ewidencji zabytków archeologicznych od strony inżynierii oprogramowania.

W rozdziale 3. znajduje się opis użytych technologii z uzasadnieniem wyboru.

Rozdział 4. zawiera opis technologiczny szkieletu aplikacji Vaadin, którego zbadanie jest celem niniejszej pracy.

Rozdział 5. opisuje proces tworzenia aplikacji oraz architekturę rozwiązania.

W rozdziale 6. jest opisany proces testowania aplikacji.

Rozdział 7. opisuje produkty, które zostały dodatkowo wytworzone w trakcie budowy aplikacji.

Rozdział 8. zawiera wnioski wyciągnięte w procesie budowy systemu.

W rozdziale 9. znajduje się podsumowanie rezultatów pracy.

% ex: set tabstop=4 shiftwidth=4 softtabstop=4 noexpandtab fileformat=unix filetype=tex spelllang=pl,en spell: